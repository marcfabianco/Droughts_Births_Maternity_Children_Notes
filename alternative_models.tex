\subsection{Alternative models and estimators}

Taking into account the current literature, I contrast different models.

\textbf{Basic TWFE}. The first is a regular TWFE model as presented in \ref{eq:twfe_basic}. The main identifying assumption is that, conditional on the fixed effects and temperatures, the remaining variation from drought intensity is exogenous and uncorrelated with other factors influencing health and birth outcomes. This model implicitly assumes parallel trends in the sense that, absent changes in drought conditions, the expected outcome evolution for groups with the same pre-treatment drought levels would have followed a common trend over time. In this setting, with two treatments that switch on and off, the counterfactual trend for each treatment group is defined relative to groups where drought conditions remain constant across periods (If my interpretation of \cite{deChaisemartin20203} is not flawed) .

\textbf{TWFE with state-specific linear time trends}. Building on the above base model, I introduce state-specific linear yearly trends. This accounts for unobserved, time-varying state-level factors that may linearly affect birth outcomes over time, ensuring that the estimated drought effects are not confounded by differential long-term trends across states.

\textbf{TWFE with state-specific linear time trends}. Building on the above base model, I introduce other sets of controls, including state-specific linear yearly trends. These trends account for heterogeneity in long-term birth outcome patterns across states, which may arise due to unobserved state-level factors such as local policies, healthcare improvements, or economic conditions evolving over time. 

\textbf{TWFE with state-specific linear time trends and seasonal controls}. As in \cite{Cohen2022}, I include a full set of municipality-by-calendar month fixed effects to control for seasonality in birth outcomes that may vary across municipalities. This allows me to account for patterns where births occurring in specific months, when temperatures or other environmental conditions tend to remain similar over time, are systematically associated with different outcomes. Combined with state-specific linear yearly trends, this specification addresses both regional long-term dynamics and local seasonal variations.

Up to here, the TWFE model relies on its standard assumptions. However, it has been shown (\cite{deChaisemartin2020}, \cite{deChaisemartin2023}) that: 1) With a single treatment turning on and off at different times, and under heterogeneous treatment effects, the TWFE estimator identifies a weighted sum of group-time treatment effects. The weights can be negative, which may distort the estimates or even reverse their sign. 2) When multiple treatments are included, the issue becomes more complex due to contamination bias. This occurs because the estimator combines comparisons between groups exposed to different treatments, which can mix the effect of one treatment with the effects of others. As a result, the coefficient on each treatment may partially reflect the effects of other treatments, introducing bias when treatment effects are heterogeneous.


\textbf{Alternative estimator (deCh-H)}. Finally, I also test the alternative estimator from \cite{deChaisemartin2023}, which is fit for my case. My case involves two mutually exclusive treatments, defined at the municipality and year-month level, that switch on and off dynamically.

The intuition of the estimator relies on identifying variation in treatment that comes from groups whose treatment changes between two periods while other treatments remain constant. By comparing these "switching" groups to others with identical treatment histories that do not change, it isolates the impact of the treatment transition. This method uses within-group and over-time variation to construct counterfactuals, focusing on more localized comparisons than standard TWFE.

The estimator relies on two assumptions: strong exogeneity and parallel trends, as defined in Assumptions 4 and 5 \cite{deChaisemartin2023}. Assumption 4 identifies treatment effects for groups whose treatment changes between consecutive periods $t-1$ and $t$, conditioning on their treatment status at $t-1$. This focuses on transitions where comparisons rely on the previous treatment state to construct counterfactuals. Assumption 5 shifts the conditioning to treatment status at $t$, expanding identification to include transitions defined by the current treatment state, rather than past status. Together, these assumptions broaden the set of transitions for which effects can be identified, including both "switching into" and "switching out of" treatment.

Parallel trends under Assumptions 4 and 5 in my setting mean that municipalities with stable drought conditions provide a counterfactual for those transitioning between treatments over consecutive months. This requires that the change in health outcomes for municipalities transitioning between drought categories matches the change observed in municipalities with stable conditions during the same period, conditional on their treatment histories. The comparison relies on consecutive-period dynamics, so counterfactual trends are constructed from groups with similar short-term treatment trajectories.


Parallel trends under Assumptions 4 and 5 hold if municipalities with the same baseline treatment status (Assumption 4) or contemporaneous treatment status (Assumption 5) would have followed similar outcome trends in the absence of treatment changes. This requires that factors influencing outcomes, such as geographical proximity, climate patterns, or economic structure, align municipalities’ responses to drought conditions. The credibility of these assumptions depends on the absence of systematic divergence in outcome trends between treated and untreated groups with comparable treatment statuses, particularly due to historical drought exposure. Including temperature and municipality-specific seasonal effects can address confounding factors that may independently influence outcomes, such as direct climatic impacts or recurring seasonal dynamics tied to economic or agricultural cycles. (Can these controls really can improve the plausibility of paralell trends assumption?)
